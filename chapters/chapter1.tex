\chapter{Introduction}
\label{chap:intro}

\lettrine[lines=3, findent=3pt, nindent=0pt]{G}{lobal} village was the term coined in the 1960s by the Canadian thinker H. Marshall McLuhan \cite{mcluhan1962} to describe the society after the birth of mass media: all parts of the World were brought together by the communication abilities given by technological advancement. \\
Nowadays, with broad adoption of digital technologies, mass-communication represents the foundation of modern society, leading to concerns about privacy and security.



%----------------------------------------------------
% PROBLEM
%----------------------------------------------------

\section{Problem}
\label{sec:problem}

The evolution of cyber-attack techniques, becoming more sophisticated than ever, poses a critical challenge in accurately and reliably detecting intrusions.\\ The first \gls{ids} was proposed in 1980 \cite{Andreson1980} and since then the industry has arisen, but modern \glsxtrshortpl{ids} aren't still perfect and suffer mainly from two problems:
\begin{itemize}
    \item[\faCaretRight] \textit{False-alarm rate}: generating too many alerts for nonthreatening situations raises the burden for security analysts, and can cause harmful attacks to be ignored;
    \item[\faCaretRight] \textit{Unknown attack detection}: attack variants and novel attacks emerge constantly and more and more frequently, hence the need of being able to detect unknown attacks. Some \gls{ids} rely on the signatures of known attacks and cannot identify new threats \cite{Hodo2017}.
\end{itemize}
Researchers in this filed aim at having a high accuracy of detection and low false alarm rate. \\
If provided with sufficient training data, machine learning algorithms can detect both attack variants and novel ones \cite{Hodo2017}. Another perk brought by machine learning models is that they do not rely heavily on domain knowledge, and so have a more lightweight design \cite{Khraisat2019}. \\ With that in mind, in order to achieve well grounded results, it is necessary to know which network features represent which attack \cite{Iglesias2015} and what data is sustainable for detecting certain types of attacks. Furthermore a comprehensive analysis of machine learning algorithms is needed for identifying the most suitable one for the implementation \cite{Liu2019}, alongside the dataset needed for its training \cite{Sharafaldin2019}.


%----------------------------------------------------
% GOAL
%----------------------------------------------------

\section{Objectives}
\label{sec:objectives}

In order to solve the problems displayed above, it would be helpful to use machine learning algorithms to improve the accuracy of threats-detection, with flexible and dynamic monitoring of the network. This project can be broken into the following milestones:

\begin{enumerate}
    \item Setting up a virtualized environment for testing purposes (this has to be scalable, to represent a realistic scenario);
    \item Choosing a \gls{sdn} Controller on which running, as an application, a monitoring software to gather data about network flow;
    \item Identifying useful traffic features and extracting them with the chosen \gls{sdn} Controller;
    \item Choosing a machine learning algorithm suitable for this use;
    \item Performing machine learning classification of the above extracted traffic features to detect malicious traffic;
    \item Evaluating the performance of the \gls{ids} using an appropriate dataset.
\end{enumerate}
Practically the first three points concern building a network monitor capable of saving traffic data. Such data will be analyzed by the software developed in the last three points.

%----------------------------------------------------
% DELIMITATION
%----------------------------------------------------

\subsection{Delimitation}
\label{subsec:delimitation}

\faEdit \quad \textbf{To be discussed with Supervisor.} \\

\textcolor{dimgray}{\lipsum[1-2]}

