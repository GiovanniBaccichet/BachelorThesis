\chapter{State of the Art}
\label{chap:state-of-the-art}

\lettrine[lines=3, findent=3pt, nindent=0pt]{T}{his} chapter rounds out the theoretical background and technologies used in the project and discussed in the following: beginning with how network traffic is made ad how to characterize malicious activities, it will then be discussed Software Defined Networking as an industry standard and then Machine Learning will be introduced, with particular emphasis on Intrusion Detection applications of Deep Learning. The reader should here be provided with sufficient knowledge to understand chapters \ref{chap:methodology}, \ref{chap:results} and \ref{chap:conclusions}.

%----------------------------------------------------
% NETWORK TRAFFIC
%----------------------------------------------------

\section{Network Traffic}
\label{sec:network-traffic}

\lipsum \\
Openflow

%----------------------------------------------------
% MALICIOUS TRAFFIC
%----------------------------------------------------

\subsection{Malicious Network Traffic}
\label{subsec:malicious-traffic}

According to \cite{Liu2019}, an \textit{intrusion} can be defined as an attempt to access information about computer systems ot to damage system operation in an illegal or unauthorized manner, hence it will be considered \textit{malicious traffic} the entirety of network traffic generated by such operations. \\
The classes of malicious traffic analyzed in this work are te following:

\begin{itemize}
    \item[\faCaretRight] \textit{Bruteforce}: this kind of attack is one of the most popular and it can be used to guess passwords or URLs (in order to discover hidden contents in web applications). It can be defined as an hint and try attack \cite{icissp18};
    \item[\faCaretRight] \textit{Botnet}: a botnet is a number of Internet connected devices, used to perform various tasks, from stealing data, to spam, or to practice \textit{Distributed Denial of Service} (DDoS) attacks \cite{icissp18} and [A];
    \item[\faCaretRight] \textit{Cross-site-scripting} (XSS):
    \item[\faCaretRight] \textit{Denial of Service} (DoS): this type of attack foresees the overloading of a network infrastructure, denying or preventing legitimate users to access resources on the system or the network \cite{Sharafaldin2019};
    \item[\faCaretRight] \textit{Distributed Denial of Service} (DDoS): POINT OUT COMMON TOOLS (EG SLOWLORIS)
    \item[\faCaretRight] \textit{Heartbleed}: this particular attack exploits a bug in the OpenSSL cryptography library (widely used implementation of the \textit{Transport Layer Security} protocol) and it can bla bla discovered in 2014 bla bla \cite{icissp18} and \cite{Carvalho2014};
    \item[\faCaretRight] \textit{Port Scanning}: this occurs when an attacker sends probe packets to gather intelligence information about the infrastructure, based on the responses received;
    \item[\faCaretRight] \textit{SQL-Injection}: malicious database queries can be used to extract bulk data from the latter; this can occur, fro example, through badly projected forms on web pages.
\end{itemize}

The dataset\footnote{See section \ref{subsec:datasets-for-evaluation}} used bla bla

%----------------------------------------------------
% TRAFFIC CHARACTERIZATION
%----------------------------------------------------

\subsection{Traffic Characterization}
\label{subsec:traffic-characterization}

See paper \cite{Iglesias2015} and  \cite{Velan2016} \\

\lipsum

%----------------------------------------------------
% SOFTWARE DEFINED NETWORKING
%----------------------------------------------------

\section{Software Defined Networking}
\label{sec:sdn}

\lipsum

%----------------------------------------------------
% MACHINE LEARNING
%----------------------------------------------------

\section{Machine Learning}
\label{sec:machine-learning}

See paper \cite{Khraisat2019} and \cite{Hodo2017} \\

\lipsum \\
Discussed in \ref{sec:machine-learning}

%----------------------------------------------------
% MACHINE LEARNING ALGORITHMS
%----------------------------------------------------

\subsection{Machine Learning Algorithms}
\label{subsec:ml-algorithms}

\lipsum

%----------------------------------------------------
% DEEP LEARNING
%----------------------------------------------------

\subsection{Deep Learning}
\label{subsec:deep-learning}

\lipsum

%----------------------------------------------------
% MACHINE LEARNING LIBRARIES
%----------------------------------------------------

\subsection{Machine Learning Libraries}
\label{subsec:ml-libraries}

\lipsum

%----------------------------------------------------
% INTRUSION DETECTION SYSTEMS
%----------------------------------------------------

\section{Intrusion Detection Systems}
\label{sec:intrusion-detection-system}

\lipsum

%----------------------------------------------------
% TAXONOMY OF INTRUSION DETECTION SYSTEMS
%----------------------------------------------------

\subsection{Taxonomy of Intrusion Detection Systems}

See paper \cite{Liu2019}

%----------------------------------------------------
% DATASETS FOR INTRUSION DETECTION SYSTEMS
%----------------------------------------------------

\subsection{Datasets for Intrusion Detection Evaluation}
\label{subsec:datasets-for-evaluation}

See paper \cite{icissp18}, \cite{Khraisat2019} and \cite{Leevy2020} \\

\lipsum[1-6]