\thispagestyle{empty}

{\bf\Huge Abstract}

\vspace{1cm}

\hfill\begin{minipage}{\dimexpr\textwidth-0.5cm}
    Cyber security has become a matter of global interest and importance. In the last couple of years, especially with the pandemic situation going on, several attacks were performed against companies and critical structures. Since almost every intrusion is put in place through the network, understanding how to identify and report malicious activity in a lightweight, reliable way is vital. \\

    This dissertation aims to use \gls{ml} techniques to detect and identify network attacks. For this purpose an \gls{ids} has been implemented and designed to be able to classify particular behaviors of network traffic with a low \textit{false-positive rate}, taking also in cosideration users' privacy, which lately has become a concern. \\ The approach used in this project differs from traditional one because of the metrics employed in the classification process: instead of inspecting the payload of each network packet, the IDS observes its \textit{flow}. In this context a \textit{flow} is a tuple of 5 values (\textit{source IP}, \textit{destination IP}, \textit{source port}, \textit{destination port}, \textit{protocol}) from which a set of useful features can be extracted and passed to the \gls{ml} model for analysis. \\ 
    
    The developed piece of software was thought to be deployed with ease in a realistic scenario. For this reason it can be interfaced with \gls{sdn} for saving and then classifying network traffic. To achieve this integration, a script for the open-source controller \textit{Ryu} has been developed: it creates a L3 switch that can generate a \texttt{.pcap} file containing the network packets that passed through it. This architecture is particularly flexible due to its modularity: the system itself can work independent of the controller chosen, which represents the only component to adapt to.\\ Almost the entirety of the system has been built using Python, since it represents a good starting point for prototyping. The \gls{ml} model, \textit{Random Forest}, was trained using a Jupyter Notebook, on the \textit{CICIDS2017} dataset. The results were satisfactory, with an accuracy of $99\%$ and an \textit{F1 score} of $99\%$.
    \end{minipage}

\vspace{1cm}

{\bf\Huge Keywords}

\vspace{1cm}

\begin{itemize}
    \itemAR Cyber Security
    \itemAR Deep Learning
    \itemAR Intrusion Detection System
    \itemAR Machine Learning
    \itemAR Software Defined Networking
\end{itemize}

\vspace{1cm}

{\bf\Huge Contact Information}

\vspace{1cm}

\faEnvelope[regular]\; \href{mailto:giovanni@baccichet.org}{\texttt{giovanni@baccichet.org}} \\

\faGlobeAmericas\; \href{https://baccichet.org}{\texttt{baccichet.org}}