\chapter{Conclusions}
\label{chap:conclusions}

\lettrine[lines=4, findent=3pt, nindent=0pt]{D}{rawing} the conclusions on this dissertation, the objectives set in \ref{sec:objectives} were achieved: the \textit{network monitor} was implemented and capable of working in both \textit{asynchronous} and \textit{synchronous} mode, in which, integrated with the \gls{sdn} controller \textit{Ryu} can collect automatically data. The metering and analysis system was built using Python and a \textit{Random Forest} classifier trained on the \textit{CICIDS2017} dataset, paying attention to the selected features to achieve enough generalization. Finally the performance has been evaluated splitting said dataset and resulted satisfactory. In addition to the objectives, on top of the \textit{core} application was also built a \gls{gui} to create a \gls{poc} of what could resemble a final product, for improving the \glsxtrshort{ux} and arrange some add-ons implementation in advance.
\par It was particularly interesting seeing how accurate the \gls{ml} model got after some tweaking of the selected features.

\section{Future work}
\label{sec:future-work}

Any future work should focus on 
\textcolor{dimgray}{\lipsum[1-3]}

\section{Final Remarks}
\label{sec:final-remarks}

\textcolor{dimgray}{\lipsum[1]}