\chapter{Conclusions}
\label{chap:conclusions}

\lettrine[lines=4, findent=3pt, nindent=0pt]{D}{rawing} the conclusions on this dissertation, the objectives set in \ref{sec:objectives} were achieved: the \textit{network monitor} was implemented and capable of working in both \textit{asynchronous} and \textit{synchronous} mode, in which, integrated with the \gls{sdn} controller \textit{Ryu} can collect automatically data. The metering and analysis system was built using Python and a \textit{Random Forest} classifier trained on the \textit{CICIDS2017} dataset, paying attention to the selected features to achieve enough generalization. Finally the performance has been evaluated splitting said dataset and resulted satisfactory. In addition to the objectives, on top of the \textit{core} application was also built a \gls{gui} to create a \gls{poc} of what could resemble a final product, for improving the \glsxtrshort{ux} and arrange some add-ons implementation in advance.
\par It was particularly interesting seeing how accurate the \gls{ml} model got after some tweaking of the selected features. Another noteworthy characteristic of the project is the adaptability achieved: it can be used in different scenarios and it can be integrated with other \gls{sdn} controller with relative ease.

\section{Future work}
\label{sec:future-work}

Any future work should focus on adopting \textit{deep learning} methodologies to improve both generalization and flexibility. More in depth, \textit{deep reinforced learning} represents an interesting solution, due to its optimization capabilities: this is a process in which an agent learns to make decisions through trial and error, in order to maximize its returns. The promise of using deep learning tools in reinforced learning is generalization:  the ability to operate correctly on previously \textit{unseen} inputs.
\par Another extension for the project would be enhancing the feature selection, including \textit{application level} data analysis would introduce new features to detect attacks that otherwise risk to blend in with the normal activity.
\par Lastly, an \gls{ips} could be added to what was realized, in order to block the detected threats and report them to the network administrator, along with some metrics concerning, for example, source host, destination host, ports and such data.

\section{Final Remarks}
\label{sec:final-remarks}

This dissertation clarified the use of \glsplural{ids}, with a particular focus on alternative traffic classification methodologies, such as \gls{ml}. As mentioned, the results were satisfactory, but further analysis on \textit{unknown} attacks would be of particular interest. The numbers achieved (\textit{accuracy} of $99.91\%$ and \textit{F1 Score} of $96.86\%$) were comparable to what was disclosed in the papers reported in chapter \ref{chap:state-of-the-art}.
\par Another remarkable fact is the modularity achieved, since it will allow for eventual integration and customization: it was particularly interesting designing the pipeline to keep the components separate and 
interchangeable, while maintaining the convenience of an \textit{all in one} system.