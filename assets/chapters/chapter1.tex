\chapter{Introduction}
\label{chap:intro}

\lettrine[lines=3, findent=3pt, nindent=0pt]{G}{lobal} village was the term coined in the 1960s by the Canadian thinker H. Marshall McLuhan \cite{mcluhan1962} to describe the society after the birth of mass media: all parts of the World were brought together by the communication abilities given by technological advancement. \\
Nowadays, with broad adoption of digital technologies, mass-communication represents the foundation of modern society, leading to concerns about privacy and security.



%----------------------------------------------------
% PROBLEM
%----------------------------------------------------

\section{Problem}
\label{sec:problem}

The evolution of cyber-attack techniques, becoming more sophisticated than ever, poses a critical challenge in accurately and reliably detecting intrusions.\\ The first \gls{ids} was proposed in 1980 \cite{Andreson1980} and since then the industry has arisen, but modern \glsxtrshortpl{ids} aren't still perfect and suffer mainly from two problems:
\begin{itemize}
    \itemAR \textit{False-alarm rate}: generating too many alerts for nonthreatening situations raises the burden for security analysts, and can cause harmful attacks to be ignored. To build an effective and reliable \gls{ids}, it is mandatory to consider the \textit{detection rate} \cite{Axelsson2000} and \cite{Liu2019};
    \itemAR \textit{Unknown attack detection}: attack variants and novel attacks emerge constantly and more and more frequently, hence the need of being able to detect unknown attacks. Some \glsplural{ids} rely on the signatures of known attacks and cannot identify new threats \cite{Hodo2017}.
\end{itemize}
Researchers in this filed aim at high detection accuracy and low false alarm rate. \\
If provided with sufficient training data, \gls{ml} algorithms can detect both attack variants and novel ones \cite{Hodo2017}. Another perk brought by \gls{ml} models is that they do not rely heavily on domain knowledge, and so have a more lightweight design \cite{Khraisat2019}. \\ With that in mind, in order to achieve well grounded results, it is necessary to know which network features represent which attack \cite{Iglesias2015} and what data is sustainable for detecting certain types of threats. Furthermore a comprehensive analysis of machine learning algorithms is needed for identifying the most suitable one for this purpose \cite{Liu2019}, alongside the dataset needed for its training \cite{Sharafaldin2019}.


%----------------------------------------------------
% GOAL
%----------------------------------------------------

\section{Objectives}
\label{sec:objectives}

As mentioned above, \gls{ml} will be used, to improve accuracy and flexibility of the \gls{ids}, in combination with a network monitor. This project can be broken into the following milestones:

\begin{enumerate}
    \item Develop a network monitor capable of saving traffic data:
    \begin{itemize}
        \itemAR Set up a virtualized environment for testing purposes;
        \itemAR Choose a \gls{sdn} Controller on which running, as application, a monitoring software to gather information about network traffic
        \itemAR Store the data collected;
    \end{itemize}
    \item Create a metering and analysis system that can classify, through \gls{ml}, the data previously acquired:
    \begin{itemize}
        \itemAR Choose a \gls{ml} algorithm suitable for this use adn train it;
        \itemAR Perform \gls{ml} classification of the above extracted traffic to detect threats to the security;
        \itemAR Evaluate the performance of the \gls{ids} using an appropriate dataset.
    \end{itemize}
\end{enumerate}

%----------------------------------------------------
% DELIMITATION
%----------------------------------------------------

\subsection{Delimitation}
\label{subsec:delimitation}

\faEdit \quad \textbf{To be discussed with Supervisor.} \\

\textcolor{dimgray}{\lipsum[1-2]}

